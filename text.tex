%-----------------------------------------------------------------------
% Document setup.

\documentclass[12pt,a4paper]{article}

\usepackage[utf8]{inputenc}
\usepackage[english,russian]{babel}

% Поля нормального размера.
\usepackage{fullpage}

% Первый абзац после заголовка раздела с отступом строки.
\usepackage{indentfirst}

\frenchspacing
\sloppy

%-----------------------------------------------------------------------
% Document body.

\begin{document}

% Half-line paragraph separation.
\setlength{\parskip}{.5\baselineskip}

% No page number at this page.
\thispagestyle{empty}

\begin{center}

Московский Государственный Университет

имени М. В. Ломоносова

Факультет вычислительной математики и кибернетики

Кафедра системного программирования

\vfill
\vfill

\Large
\textbf{
Распознавание именованных сущностей \linebreak
с использованием алгоритмов машинного обучения, \linebreak
основанных на принципе минимальной длины описания
}

\large
\hspace{0pt}
\linebreak
\textit{курсовая работа}
\normalsize

\vfill
\vfill

\begin{flushright}
Выполнил:

Студент NNN-й группы

Шалаев Михаил Михайлович

% Просто пустой абзац.
\hspace{0pt}

Научный руководитель:

Белобородов Иван Борисович
\end{flushright}

\vfill

Москва \\
2013
\end{center}

\clearpage
\tableofcontents

\clearpage
\section{Введение}

\clearpage
\section{Обзор предметной области}

\clearpage
\section{Проектирование}

\clearpage
\section{Описание системы}

\clearpage
\section{Заключение}

\clearpage
\begin{thebibliography}{9}

% Чтобы список литературы был в содержании.
\addcontentsline{toc}{section}{Список литературы}

\bibitem{collins2002a}
Collins, M., 2002.
Ranking Algorithms for Named–Entity Extraction:
Boosting and the Voted Perceptron.
\textit{Proceedings of the 40th Annual Meeting on Association for
Computational Linguistics}, pp. 489–496.

\bibitem{collins2002b}
Collins, M., 2002.
Discriminative training methods for hidden Markov models:
theory and experiments with perceptron algorithms.
\textit{Proceedings of the ACL-02 conference on Empirical methods in
natural language processing}, pp. 1–8.

\bibitem{lafferty2001}
Lafferty, J., McCallum, A. and Pereira, F., 2001.
Conditional Random Fields: Probabilistic Models for Segmenting and
Labeling Sequence Data. \textit{Proceedings of the Eighteenth
International Conference on Machine Learning}, pp. 282–289.

\bibitem{mccallum2000}
McCallum, A., Freitag, D. and Pereira, F., 2000.
Maximum Entropy Markov Models for Information Extraction and
Segmentation. \textit{Proceedings of the Seventeenth International
Conference on Machine Learning}, pp. 591–598.

\bibitem{nadeua2007}
Nadeau, D. and Sekine, S., 2007.
A survey of named entity recognition and classification.
\textit{Lingvisticae Investigationes}, 30 (1), pp. 3–26.

\bibitem{ratinov2009}
Ratinov, L. and Roth, D., 2009.
Design Challenges and Misconceptions in Named Entity Recognition.
\textit{Proceedings of the Thirteenth Conference on Computational
Natural Language Learning}, pp. 147–155.

\bibitem{rissanen1983}
Rissanen, J., 1983.
A Universal Prior for Integers and Estimation by Minimum Description
Length. \textit{The Annals of Statistics}, 11 (2), pp. 416–431.

\end{thebibliography}

\end{document}

%-----------------------------------------------------------------------
